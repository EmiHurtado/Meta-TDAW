\documentclass[10pt,a4paper]{article}
\usepackage[left=2cm,right=2cm,top=2cm,bottom=2cm]{geometry}
\usepackage[dvipsnames]{xcolor}
\usepackage[fleqn]{mathtools}
\usepackage{booktabs}
\usepackage{amsmath}
\usepackage{latexsym}
\usepackage{graphicx}
\usepackage{nccmath}
\usepackage{multicol}
\usepackage{listings}
\usepackage{tasks}
\usepackage{color}
\usepackage{float}
\usepackage{lipsum}

\definecolor{colorIPN}{rgb}{0.5, 0.0,0.13}
\definecolor{colorESCOM}{rgb}{0.0, 0.5,1.0}
\graphicspath{ {imagenes/} }

\begin{document}
%#########################################################
\begin{titlepage}
	\centering
	{ \huge \bfseries \color{colorIPN}{Instituto Politécnico Nacional} \par}
	{ \Large \bfseries  \color{colorESCOM}{Escuela Superior de Cómputo} \par }
	\vspace{1cm}
	{\huge\Large \color{colorIPN}{Tecnologías para Desarrollo de Aplicaciones Web}.\par}
	\vspace{1.5cm}
	{\huge\Large  \color{colorESCOM}{Ejercicio 1 : Tablas y Formularios}\par}
		\vspace{2cm}
	{\Large\itshape \color{colorIPN}{Profesor: M. en C. José Asunción Enríquez Zárate}\par} \hfill \break
	\vspace{2cm}
	{\Large\itshape \color{colorIPN}{Alumno: Hurtado Morales Emiliano}\par} \hfill \break
	{\Large\itshape \color{colorIPN}{ehurtadom1700@alumno.ipn.mx}\par} \hfill \break
	{\Large\itshape \color{colorIPN}{4CM3} \par}
	\vfill
	{\large \color{colorIPN}{\today}\par} 
	\vfill
\end{titlepage}

\renewcommand\lstlistingname{Quelltext} 

\lstset{ 
	language=Java,
	basicstyle=\small\sffamily,
	numbers=left,
	numberstyle=\tiny,
	frame=tb,
	tabsize=4,
	columns=fixed,
	showstringspaces=false,
	showtabs=false,
	keepspaces,
	commentstyle=\color{Violet},
	keywordstyle=\color{colorIPN} \bfseries,
	stringstyle=\color{colorESCOM}
}

\settasks{
	counter-format=(tsk[r]),
	label-width=4ex
}

\tableofcontents 
\pagebreak

\listoffigures
\pagebreak

%################################################
\section{\color{colorIPN}{Introducción}}
HTML5 es la última versión de la tecnología HTML, cuyas siglas corresponden a “HyperText Markup Language”, que tiene el siguiente significado:

\begin{itemize}
	\item  HyperText, cuyo significado es hipertexto, que no es más que un texto que enlaza con otros contenidos, que pueden ser otro texto u otro archivo. Esto es la base del funcionamiento de la web tal y como la conocemos, que no es más que páginas y recursos interconectados.
	\item  Markup, que significa marca o etiqueta, ya que todas las páginas web están construidas en base a etiquetas, desde las primeras versiones hasta las últimas etiquetas de HTML5. Un ejemplo de una etiqueta HTML es la que identifica a un párrafo, que se compone de la etiqueta, el contenido de la etiqueta y el cierre del párrafo.
	\item Languaje, cuyo significado es lenguaje, porque HTML es un lenguaje, es decir, tiene sus normas, tiene su estructura y una serie de convenciones que nos sirven para definir tanto la estructura como el contenido de una web.
	
\end{itemize}

\subsection{ \color{colorESCOM}{Formularios}}
Un formulario HTML es una sección de un documento que contiene contenido normal, código, elementos especiales llamados controles (casillas de verificación (checkboxes), radiobotones (radio buttons), menúes, etc.), y rótulos (labels) en esos controles. Los usuarios normalmente "completan" un formulario modificando sus controles (introduciendo texto, seleccionando objetos de un menú, etc.), antes de enviar el formulario a un agente para que lo procese (p.ej., a un servidor web, a un servidor de correo, etc.)

Los usuarios interaccionan con los formularios a través de los llamados controles.

\subsection{ \color{colorESCOM}{Tablas}}
Una tabla es un conjunto estructurado de datos distribuidos en filas y columnas (datos tabulados). Una tabla permite buscar con rapidez y facilidad valores entre diferentes tipos de datos que indiquen algún tipo de conexión.

El aspecto básico de una tabla es que es un elemento rígido. Es fácil interpretar la información haciendo asociaciones visuales entre los encabezados de las filas y las columnas.

\subsection{ \color{colorESCOM}{Trabajo}}
El proyecto presentado a continuación es una unión de los formularios y tablas para realizar una solicitud de empleo en línea, empleando cada funcionalidad que provee estas herramientas de HTML.

\pagebreak

%################################################
\section{\color{colorIPN}{Conceptos}}

\begin{itemize}
	\item  HTML: Hace referencia al lenguaje de marcado para la elaboración de páginas web.
	\item  Formulario: Documento, físico o digital, elaborado para que un usuario introduzca datos estructurados en las zonas correspondientes, para ser almacenados y procesados posteriormente.
	\item  Tabla: Herramienta de organización de información que se utiliza en bases de datos en la informática.
	\item Botón: Pieza que se pulsa o se hace girar para activar, interrumpir o regular el funcionamiento de un aparato o de un mecanismo.
	\item Celda: Espacio o campo en donde se introducen los datos.
	\item Fila: Serie ordenada de individuos o elementos que se disponen en línea horizontal.
	\item Columna: Serie ordenada de individuos o elementos que se disponen en línea vertical.
	\item Acción: Realización de un acto o hecho, o el efecto que produce determinado hecho en cuestión. 
	
\end{itemize}

\pagebreak

%################################################
\section{\color{colorIPN}{Desarrollo}}
En este apartado, se hará un seguimiento por el proceso de creación de la página web "Solicitud de Empleo", publicada en la URL: https://upbeat-swartz-2f3e65.netlify.app/

%################################################
\subsection{
	\textit{
		\color{colorESCOM}{Creación}
	}
}
Se trabajó con el IDE Visual Studio Code para una mejor visualización del código y mayor facilidad de programar.
Primero, se usó la plantilla que provee VSC para desarrollar una página web.

\begin{figure}[H]
	\includegraphics[scale=.54]{Capture}
	\centering
	\caption{Plantilla de VSC}
	\label{img:Capture}
\end{figure} 

Se definió como título "Solicitud de Empleo" y se empezó a codificar la página.

\begin{figure}[H]
	\includegraphics[scale=.54]{Capture2}
	\centering
	\caption{Fieldset, Formulario y Tabla}
	\label{img:Capture2}
\end{figure} 

Primero, se definió un fieldset de un ancho de 1000px, para que la información se viera de la mejor forma distribuida. Posteriormente, se creó un formulario con la instrucción form, donde se estableció como acción "solicitudEmpleo" y como método "POST". Estas dos características no se emplearon en este trabajo, pero sirven como muestra del funcionamiento de un formulario. De igual forma, se le adjuntó un id y name.

Luego, se inició la construcción de una tabla para una mayor facilidad de acomodo de toda la información requerida. Se le puso un borde de 1 y una anchura de 1000.

De esta forma, se pasó a definir todas sus filas, columnas y celdas. Para un mayor dinamismo a la hora de construir la solicitud, se decantó por realizar una tabla para cada apartado del documento. Asimismo, se emplearon distintas funcionalidades del input que se explicarán a continuación.

\subsection{
	\textit{
		\color{colorESCOM}{Funcionalidades del Formulario}
	}
}

Se emplearon siete herramientas para la creación del formulario, la mayoría con la instrucción input:

\begin{itemize}
	\item  text: Se le asignó a cada uno un id y name relacionado a lo requisitado en ese punto. También, se estableció en la mayoría una extensión máxima de 20, o hasta 50, carácteres con la instrucción maxlength; se pintó una oración de fondo en varios inputs para facilitar el saber que poner en cada uno de ellos con placeholder y se definió que fuera obligatorio el llenado de cada espacio con required.
	
	\begin{figure}[H]
		\includegraphics[scale=.54]{Capture4}
		\centering
		\caption{text}
		\label{img:Capture4}
	\end{figure} 
	
	\item  number: Ya definidos el id y name, se le dió una extensión máxima de 20 carácteres, un valor predefinido de 18 (value) y un rango entre 18 y 45 (min - max), haciendo caso a las reglas de negocio preestablecidas. Por último, se pidió que fuera obligatorio su llenado.
	
	\begin{figure}[H]
		\includegraphics[scale=.54]{Capture5}
		\centering
		\caption{number}
		\label{img:Capture5}
	\end{figure} 
	
	\item  date: Sólamente se le asignó name, id y que fuera necesaria la elección de una fecha.
	
	\begin{figure}[H]
		\includegraphics[scale=.54]{Capture3}
		\centering
		\caption{date}
		\label{img:Capture3}
	\end{figure} 
	
	\pagebreak	
	
	\item  radio: El uso de radio permite elegir entre un conjunto de opciones, sólo permitiendo quedarse con una. Se le asignó un name al conjunto de elecciones y un id diferente a cada uno. También, se le adjuntó un value por simple formalidad. Como característica adicional, se empleo la instrucción label para unir una palabra a cada radio.
	
	\begin{figure}[H]
		\includegraphics[scale=.54]{Capture6}
		\centering
		\caption{radio}
		\label{img:Capture6}
	\end{figure} 
	
	\item  checkbox: Muy parecido al radio, el tipo checkbox permite elegir varias opciones entre un conjunto de ellas. Se le asignó un name al grupo y un id individual. También, se escribió un value y el label para saber a que se refiere cada checkbox.
	
	\begin{figure}[H]
		\includegraphics[scale=.54]{Capture7}
		\centering
		\caption{checkbox}
		\label{img:Capture7}
	\end{figure} 
	
	\item  textarea: Esta herramienta es un cuadro de texto, sólo que se puede modificar su tamaño. Se definió con 7 filas y 100 columnas por pura estética, aunque puede ser moldeado por el candidato. De igual modo, se le asignó un name.
	
	\begin{figure}[H]
		\includegraphics[scale=.54]{Capture9}
		\centering
		\caption{textarea}
		\label{img:Capture9}
	\end{figure} 
	
	\item  button (submit y reset): Con estas intrucciones, se puede crear botones que realizan distintas acciones. En el caso de submit, manda toda la información a la base de datos enlazada o al procesamiento requisitado (En esta práctica, no se trabajó en esa funcionalidad). En cuanto a reset, borra toda la información en cada uno de los apartados del formulario.
	
	\begin{figure}[H]
		\includegraphics[scale=.54]{Capture8}
		\centering
		\caption{button. submit y reset}
		\label{img:Capture8}
	\end{figure} 
	
\end{itemize}

\subsection{
	\textit{
		\color{colorESCOM}{Funcionalidades de la Tabla}
	}
}

Con respecto a la tabla, más alla del uso de las filas y columnas, se empleó su capacidad de unir estas mismas con colspan y rowspan. Estas instrucciones fueron usadas cuanto fueron necesarias, de forma que se consiguiera el formato pedido.

\begin{figure}[H]
	\includegraphics[scale=.54]{Capture11}
	\includegraphics[scale=.54]{Capture12}
	\centering
	\caption{colspan y rowspan}
	\label{img:Capture12}
\end{figure} 

Asimismo, se emplearon nbsp y br para darle mayor estilo a la solicitud, al hacer espacios y saltos de línea, respectivamente.

\pagebreak

%################################################
\section{\color{colorIPN}{Resultados}}

Pasando a los resultado, se mostrará la página web y las distintas funcionalidades que provee, la cual se puede encontrar aquí: https://upbeat-swartz-2f3e65.netlify.app/.
%################################################
\subsection{	\color{colorESCOM}{Página Web}}

\begin{figure}[H]
	\includegraphics[scale=.54]{Capture10}
	\centering
	\caption{Funcionalidades}
	\label{img:Capture10}
\end{figure} 

En esta imagen, es posible observar la mayoría de funcionalidades comentadas en el apartado de desarrollo. Se tiene el fieldset y la separación en tablas por cuestión de estilo, el ingreso de datos por texto, número, fecha, radio y checkbox. Asimismo, esta el uso de colspan y rowspan para modificar la visibilidad de la información y el uso de bgcolor para sombrear ciertas celdas.

\begin{figure}[H]
	\includegraphics[scale=.54]{Capture13}
	\centering
	\caption{Funcionalidad. textarea}
	\label{img:Capture13}
\end{figure} 

\begin{figure}[H]
	\includegraphics[scale=.54]{Capture14}
	\centering
	\caption{Funcionalidad. sumbit y reset}
	\label{img:Capture14}
\end{figure} 

Al igual, se tiene el textarea y los botones de registrar y cancelar.

\pagebreak

%################################################
\section{\color{colorIPN}{Conclusión}}
HTML es un lenguaje de hipertexto que abre varias posibilidades para crear páginas webs. Ciertamente, hoy en día ya hay tecnologías mucho más avanzadas para el desarrollo web, pero siempre es importante entender los principios de ciertas herramientas. Además, el trabajar con lo simple permite ir subiendo de dificultad poco a poco y solidificando cada vez más el conocimiento.

Considero que la práctica permite un aprendizaje prioritario y hace ver la estructura del código finamente.

\color{colorIPN}{
	\begin{flushright}
		\textit{
			Hurtado Morales Emiliano
		}
	\end{flushright} \hfill \break
}

\pagebreak

%################################################

\section{\color{colorIPN}{Referencias Bibliográficas}}
\color{colorESCOM}{
	\begin{thebibliography}{10}
		\bibitem[HTMLQuick]{HTMLQuick}
		De León, D.
		\newblock {\em Tablas en HTML}
		\newblock Recuperado el 4 de marzo de 2022, de https://www.htmlquick.com/es/tutorials/tables.html
	
		\bibitem[Conclase]{Conclase}
		Conclase
		\newblock {\em Formularios en documentos HTML}
		\newblock Recuperado el 4 de marzo de 2022, de http://html.conclase.net/w3c/html401-es/interact/forms.html
	
		\bibitem[OpenWebinars, 2019]{OpenWebinars}
		OpenWebinars
		\newblock {\em Que es HTML5}
		\newblock Recuperado el 4 de marzo de 2022, de https://openwebinars.net/blog/que-es-html5/
	\end{thebibliography}
}

\end{document}
